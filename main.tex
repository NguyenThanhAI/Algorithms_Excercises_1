\documentclass[14pt, a4paper]{article}
\usepackage{minitoc}
\usepackage[left=3.00cm, right=2.5cm, top=2.00cm, bottom=2.00cm]{geometry}
\usepackage{amsmath}
\usepackage{amssymb}
\usepackage{amsthm}
\usepackage{mathtools}
\usepackage{graphicx}
%\usepackage{algpseudocode}
%\usepackage{algorithm}
\usepackage[ruled,vlined,linesnumbered]{algorithm2e}
\usepackage{blindtext}
\usepackage{setspace}
\usepackage[utf8]{inputenc}
\usepackage[utf8]{vietnam}
\usepackage[center]{caption}
\usepackage[shortlabels]{enumitem}
\usepackage{fancyhdr} % header, footer
\usepackage{hyperref} % loại bỏ border với mục lục và công thức
\usepackage[nonumberlist, nopostdot, nogroupskip]{glossaries}
\usepackage{glossary-superragged}
\usepackage{tikz,tkz-tab}
\setglossarystyle{superraggedheaderborder}
\pagestyle{fancy}
%\usepackage[style=numeric,sortcites]{biblatex}
%\addbibresource{ref.bib}
%\usepackage[numbers]{natbib}
\usepackage{indentfirst}
\usepackage[natbib,backend=biber,style=ieee, sorting=ynt]{biblatex}
\bibliography{ref.bib}

\graphicspath{{./figures/}}

\fancyhf{}
\rhead{\textbf{Môn học: Toán rời rạc và thuật toán}}
\lhead{\textbf{GVHD: PGS. TS. Nguyễn Thị Hồng Minh}}
\rfoot{\thepage}
\lfoot{\textbf{Nhóm học viên thực hiện: Nhóm 01 - Cao học Khoa học dữ liệu - K4}}
\renewcommand{\headrulewidth}{0.4pt}
\renewcommand{\footrulewidth}{0.4pt}
%
%\numberwithin{equation}{section}
%\numberwithin{algorithm}{section}
%\numberwithin{figure}{section}
%
%\setlength{\parindent}{0.5cm}
%
%\setcounter{secnumdepth}{3} % Cho phép subsubsection trong report
%\setcounter{tocdepth}{3} % Chèn subsubsection vào bảng mục lục

%\newtheorem{dl}{Định lý}
%\newtheorem{md}{Mệnh đề}
%\newtheorem{bd}{Bổ đề}
%\newtheorem{dn}{Định nghĩa}
%\newtheorem{hq}{Hệ quả}

%\newtheorem{baitap}{Bài tập}
%\newtheorem*{loigiai}{Lời giải}

%\numberwithin{dl}{section}
%\numberwithin{md}{section}
%\numberwithin{bd}{section}
%\numberwithin{dn}{section}
%\numberwithin{hq}{section}

\setlength{\parindent}{0cm}

\newtheorem{dl}{Định lý}
\newtheoremstyle{sltheorem}
{}                % Space above
{}                % Space below
{\normalfont}        % Theorem body font % (default is "\upshape")
{}                % Indent amount
{\bfseries}       % Theorem head font % (default is \mdseries)
{.}               % Punctuation after theorem head % default: no punctuation
{ }               % Space after theorem head
{}                % Theorem head spec
\theoremstyle{sltheorem}
\newtheorem{baitap}{Bài tập}
\newtheoremstyle{soltheorem}
{}                % Space above
{}                % Space below
{\normalfont}        % Theorem body font % (default is "\upshape")
{}                % Indent amount
{\bfseries}       % Theorem head font % (default is \mdseries)
{.}               % Punctuation after theorem head % default: no punctuation
{\newline}               % Space after theorem head
{}                % Theorem head spec
\theoremstyle{soltheorem}
\newtheorem*{loigiai}{Lời giải}

\onehalfspacing

\begin{document}

    \begin{titlepage}

        \newcommand{\HRule}{\rule{\linewidth}{0.5mm}} % Defines a new command for the horizontal lines, change thickness here

        \center % Center everything on the page

        %----------------------------------------------------------------------------------------
        %	HEADING SECTIONS
        %----------------------------------------------------------------------------------------
        \textsc{\LARGE Đại học Quốc Gia Hà Nội}\\[0.5cm]
        \textsc{\LARGE Trường đại học Khoa học tự nhiên}\\[0.5cm] % Name of your university/college
        \textsc{\LARGE Khoa Toán - Cơ - Tin học}\\[0.5cm]

        \includegraphics[scale=0.2]{HUS-logo.jpg}\\[0.5cm]

        \textsc{\Large Chuyên ngành: Khoa học dữ liệu}\\[0.5cm] % Major heading such as course name


        %----------------------------------------------------------------------------------------
        %	TITLE SECTION
        %----------------------------------------------------------------------------------------

        \HRule \\[0.4cm]
        { \huge \bfseries Bài tập môn học}\\[0.4cm] % Title of your document
        \HRule \\[1.5cm]

        \textsc{\Large Môn học: Toán rời rạc và thuật toán}\\[1cm] % Minor heading such as course title


        \textsc{\Large Bài tập 1: Các khái niệm cơ bản về thuật toán \\ và các phương pháp thiết kế thuật toán}\\[1cm]


        %----------------------------------------------------------------------------------------
        %	AUTHOR SECTION
        %----------------------------------------------------------------------------------------
        \begin{minipage}{0.4\textwidth}
            \begin{flushleft} \large
            \emph{Giảng viên hướng dẫn:} \\
            PGS. TS. Nguyễn Thị Hồng Minh % Supervisor's Name
            \end{flushleft}
        \end{minipage}\\[0.5cm]

        \begin{minipage}{0.4\textwidth}
        \begin{flushleft} \large
        \emph{Nhóm học viên thực hiện:}\\
        Nguyễn Chí Thanh \\
        MSHV: 21007925 \\ % Your name
        Vũ Ngọc Đại \\
        MSHV: 21007977 \\
        Vũ Minh Hưng \\
        MSHV: 21007973 \\
        Lê Diệu Thúy \\
        MSHV: 21007922 \\
        Lớp: Khoa học dữ liệu - K4
        \end{flushleft}
        \end{minipage}


        % If you don't want a supervisor, uncomment the two lines below and remove the section above
        %\Large \emph{Author:}\\
        %John \textsc{Smith}\\[3cm] % Your name

        %----------------------------------------------------------------------------------------
        %	DATE SECTION
        %----------------------------------------------------------------------------------------

        % I don't want day because it is English
        % {\large \today}\\[2cm] % Date, change the \today to a set date if you want to be precise

        %----------------------------------------------------------------------------------------
        %	LOGO SECTION
        %----------------------------------------------------------------------------------------

        %\includegraphics{logo/rsz_3logo-khtn.png}\\[1cm] % Include a department/university logo - this will require the graphicx package

        %----------------------------------------------------------------------------------------

        \vfill % Fill the rest of the page with whitespace

    \end{titlepage}

    \nocite{*}

    \newpage

    \begin{algorithm}
        \DontPrintSemicolon % Some LaTeX compilers require you to use \dontprintsemicolon instead
        \KwIn{A finite set $A=\{a_1, a_2, \ldots, a_n\}$ of integers}
        \KwOut{The largest element in the set}
        $max \gets a_1$\;
        \For{$i \gets 2$ \textbf{to} $n$} {
          \If{$a_i > max$} {
            $max \gets a_i$\;
          }
        }
        \Return{$max$}\;
        \caption{{\sc Max} finds the maximum number}
        \label{algo:max}
    \end{algorithm}

    \begin{algorithm}
        \DontPrintSemicolon % Some LaTeX compilers require you to use \dontprintsemicolon instead 
        \KwIn{A set $C = \{c_1, c_2, \ldots, c_r\}$ of denominations of coins, where $c_i > c_2 > \ldots > c_r$ and a positive number $n$}
        \KwOut{A list of coins $d_1,d_2,\ldots,d_k$, such that $\sum_{i=1}^k d_i = n$ and $k$ is minimized}
        $C \gets \emptyset$\;
        \For{$i \gets 1$ \textbf{to} $r$}{
          \While{$n \geq c_i$} {
            $C \gets C \cup \{c_i\}$\;
            $n \gets n - c_i$\;
          }
        }
        \Return{$C$}\;
        \caption{{\sc Change} Makes change using the smallest number of coins}
        \label{algo:change}
    \end{algorithm}

    \begin{algorithm}
        \DontPrintSemicolon % Some LaTeX compilers require you to use \dontprintsemicolon instead
        \KwIn{A sequence of integers $\langle a_1, a_2, \ldots, a_n \rangle$}
        \KwOut{The index of first location witht he same value as in a previous location in the sequence}
        $location \gets 0$\;
        $i \gets 2$\;
        \While{$i \leq n$ \textbf{and} $location = 0$}{
          $j \gets 1$\;
          \While{$j < i$ \textbf{and} $location = 0$}{
            % The "u" before the "If" makes it so there is no "end" after the statement, so the else will then follow
            \uIf{$a_i = a_j$}{
              $location \gets i$\;
            }
            \Else{
              $j \gets j + 1$\;
            }
          }
          $i \gets i + 1$\;
        }
        \Return{location}\;
        \caption{{\sc FindDuplicate}}
        \label{algo:duplicate}
    \end{algorithm}

    \begin{algorithm}
        \DontPrintSemicolon
        \KwIn{A sequence of integers $\langle a_1, a_2, \ldots, a_n \rangle$}
        \KwOut{The index of first location witht he same value as in a previous location in the sequence}
        $location \gets 0$\;
        $i \gets 2$\;
        \While{$i \leq n \land location = 0$}{
          $j \gets 1$\;
          \While{$j < i \land location = 0$}{
            % The "l" before the If makes it so it does not expand to a second line
            \lIf{$a_i = a_j$}{
              $location \gets i$\;
            }
            \lElse{
              $j \gets j + 1$\;
            }
          }
          $i \gets i + 1$\;
        }
        \Return{location}\;
        \caption{{\sc FindDuplicate2}}
        \label{algo:duplicate2}
    \end{algorithm}

    \begin{algorithm}
        \DontPrintSemicolon
        \KwIn{Dãy các số nguyên $A=\lbrace a_1, a_2, \dots, a_n \rbrace$}
        \KwOut{Dãy con có trọng lượng lớn nhất}

        $s_1 \gets a_1$\;
        $e_1 \gets a_1$\;
        \For{$i \gets 2$ \textbf{to} $n$} {
            $e_i \gets \max(e_{i-1}, e_{i-1} + a_i)$\;
            $s_i \gets \max(s_{i-1}, e_i)$\;
        }
        \Return{$s_n$}\;
        \caption{Thuật toán tìm dãy con có trọng lượng lớn nhất}
    \end{algorithm}

    \newpage
    \printbibliography[title={TÀI LIỆU THAM KHẢO}]

\end{document}